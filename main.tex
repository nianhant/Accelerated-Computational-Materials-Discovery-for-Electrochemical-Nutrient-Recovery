\documentclass[journal=jacsat,manuscript=article]{achemso}

\usepackage[version=4]{mhchem}
\usepackage{graphicx}
\usepackage{subcaption}
\usepackage{siunitx}
\usepackage{xr}
\usepackage{hyperref} 
\usepackage{cleveref}

\externaldocument[prefix]{output}

\captionsetup[subfigure]{skip=1pt,singlelinecheck=false}
\newcommand*\mycommand[1]{\texttt{\emph{#1}}}
\author{Nianhan Tian}
\affiliation[Georgia Institute of Technology]
{School of Chemical and Biomolecular Engineering, Georgia Institute of Technology, Atlanta, Georgia 30318 USA}

\author{Ehsan Abbasi}
\affiliation[Texas Tech University]
{Department of Chemical Engineering, Texas Tech University, Lubbock, Texas 79409 USA}

\author{Haldrian Iriawan}
\affiliation[Massachusetts Institute of Technology]
{Department of Materials Science \& Engineering, Massachusetts Institute of Technology, Cambridge, Massachusetts 02139 USA}

\author{Paul Kohl}
\affiliation[Georgia Institute of Technology]
{School of Chemical and Biomolecular Engineering, Georgia Institute of Technology, Atlanta, Georgia 30318 USA}

\author{Gerardine G. Botte}
\affiliation[Texas Tech University]
{Department of Chemical Engineering, Texas Tech University, Lubbock, Texas 79409 USA}

\author{Yang Shao-Horn}
\affiliation[Massachusetts Institute of Technology]
{Department of Materials Science & Engineering, Massachusetts Institute of Technology, Cambridge, Massachusetts 02139 USA}

\author{Andrew J. Medford}
\affiliation[Georgia Institute of Technology]
{School of Chemical and Biomolecular Engineering, Georgia Institute of Technology, Atlanta, Georgia 30318 USA}
\email{ajm@gatech.edu}

\title{Electrochemical stability of metal and oxide catalysts in the presence of nitrogen ligands}

\abbreviations{IR,NMR,UV}
\keywords{American Chemical Society, \LaTeX}

%%%%%%%%%%%%%%%%%%%%%%%%%%%%%%%%%%%%%%%%%%%%%%%%%%%%%%%%%%%%%%%%%%%%%
%% The manuscript does not need to include \maketitle, which is
%% executed automatically.
%%%%%%%%%%%%%%%%%%%%%%%%%%%%%%%%%%%%%%%%%%%%%%%%%%%%%%%%%%%%%%%%%%%%%
\begin{document}
%%%%%%%%%%%%%%%%%%%%%%%%%%%%%%%%%%%%%%%%%%%%%%%%%%%%%%%%%%%%%%%%%%%%%
%% The abstract environment will automatically gobble the contents
%% if an abstract is not used by the target journal.
%%%%%%%%%%%%%%%%%%%%%%%%%%%%%%%%%%%%%%%%%%%%%%%%%%%%%%%%%%%%%%%%%%%%%
\begin{abstract}
\end{abstract}

\section{Introduction}
Motivation
Electrocatalytic nitrogen conversion in waste-sludge environments (0.1 M KOH + 0.1 M glycine) for ammonia formation.

The challenge: balancing activity with catalyst stability under strongly oxidizing/reducing conditions.

Prior work: Pourbaix diagrams indicate instability of common transition metals under applied potentials.

Objective: Identify catalysts that maintain stability while promoting selective nitrogen conversion.

Scope of this study
Comparative analysis of Au, Pd, AuPd, Ni, Cu, TiNi, TiCu electrodes at 2 V vs RHE.

Integration of electrochemical data (CV, charge passed), product quantification (\ce{NO2-}/\ce{NO3-}/\ce{NH3}), and Pourbaix-based stability predictions.


\section{Methods}
\subsection{Pourbaix Diagram Construction}
Pourbaix diagrams were constructed to map the thermodynamic stability regions of bulk metals/alloys and their metal-ligand complexes in the presence of nitrogen-containing ligands at concentrations in the range expected to be present in EWAS.
%under electrolysis of waste activated sludge (EWAS) potentials in basic environment. 

Free energies of bulk metals were obtained from the Materials Project database \cite{Jain2013TheInnovation}, based on generalized gradient approximation (GGA) density functional theory (DFT) calculations. Free energies for aqueous ligands and metal-ligand complexes were calculated from equilibrium constants sourced from standard thermochemical handbooks \cite{Wagman1982TheUnits, Smith1989CriticalConstants, Bard2017StandardSolution, Bjerrum1957StabilitySubstances} and literature \cite{Meng1996PrinciplesReview, Azadi2019DataComplexes, Aviles2022ExploringNH3, Oraby2023SelectiveSolutions, Harrington2005DeterminationIon}. We focus here on metal elements that are commonly used as electrode materials and those where metal-ligand complex stabilities are available: Au, Cu, Ni, Co, Mg, Mn, Ti, Zn, Ag, Cd, Sr, Pt, Pd, and Fe. The aqueous ligands considered in this study include \ce{NH3-}, glycine, and \ce{CN-}. The targeted experimental potential ranges from -2.0 to 2.3 V (vs RHE) and the pH ranges from 11.5 to 13.5, but Pourbaix diagrams are constructed over a broader range for reference.


The diagrams were constructed based on the framework introduced by previous studies \cite{PourbaixAtlasSolutions, Huang2017ImprovedCompounds,Huang2015ElectrochemicalCalculations,Singh2017ElectrochemicalMaterials,Patel2019EfficientCompounds,Persson2012PredictionStates, Ding2018ElectrochemicalStates, Thompson2011PourbaixSystems} at 298.15 K. All reaction species, i.e., metals, alloys, oxides, aqueous ions and ligands, and water, are connected by their corresponding redox or acid-base reactions. Then the reaction chemical potentials are derived from equilibrium relationships between species, using the Nernst equation as a function of applied potential and pH.

For a general redox reaction:

\begin{equation} \label{eq:reaction}
aA + h\text{H}^+ + z\text{e}^- \leftrightarrow bB + \text{H}_2\text{O}
\end{equation}

The Nernst equation is:

\begin{equation} \label{eq:nernst}
E = E^\circ - \frac{k}{z} \log \left(\frac{[A]^a}{[B]^b}\right) - \frac{k \cdot h}{z} \, \text{pH},
\end{equation}

where \( E \) is the cell potential under non-standard conditions, \( E^\circ \) is the standard electrode potential, and \( k = \frac{RT}{F \ln 10} \). \( A \) and \( B \) are the activities of the reactants and products, and \( z \) is the number of electrons transferred. \( a \), \( b \), and \( h \) are the stoichiometric coefficients of the reactants, products, and protons, respectively.

Aqueous ion activities are approximated to be their concentration and are assumed to be \SI{1e-4}M \cite{Huang2017ImprovedCompounds, Wang2020PredictingFunctional, Patel2019EfficientCompounds, Thompson2011PourbaixSystems}. For ligands such as \ce{NH3}, glycine, and \ce{CN^-}, we apply a fixed representative concentration for each species based on experimentally relevant conditions. These concentrations are held constant throughout the Pourbaix analysis and are noted in the captions of each diagram. This approach allows for direct comparison with experimental systems and avoids introducing concentration-dependent variability into the stability trends.

For alloy Pourbaix diagrams involving multiple elements, all candidate alloy phases are generated from valid stoichiometric combinations of the relevant elemental entries available from the Materials Project \cite{Jain2013TheInnovation}. These combinations must satisfy the compositional constraint specific to the system under consideration (e.g. Ni:Ti = 1:1) \cite{Thompson2011PourbaixSystems}. To manage the computational complexity of this enumeration, the number of alloying elements is restricted to two. This enables systematic exploration of alloy stability within the compositional space relevant to each Pourbaix diagram while maintaining tractable scaling. 
% It is important to note that different compositional constraints can yield different Pourbaix diagrams, as the relevant free energies can vary depending on the reference stoichiometry.



The Pourbaix diagrams were generated by evaluating the chemical potentials of all species across a finely spaced potential-pH grid (4000 $\times$ 4000 points) and identifying the most thermodynamically stable species at each grid point. The thermodynamic stability of each species was calculated by comparing chemical potentials at each grid point across the potential-pH space. Stability regions were identified for bulk metals, aqueous ions, and metal-ligand complexes. 


\subsection{Incorporation of Nitrogen Ligands}

To extend the Pourbaix framework to systems containing nitrogen-based ligands (\ce{NH3}, glycine, and \ce{CN-}), ligand concentrations are represented as pLigand in \ref{eq:pligand} and incorporated into pH-dependent mass balance equations, assuming that the sum of the concentrations of protonated and deprotonated species is kept constant. For example, the dissociation equilibrium of \ce{NH3} (\ref{eq:NH3 equilibrium}) indicates the relationship between pH and p\ce{NH3} via \ref{eq:pNH3}:

\begin{equation} \label{eq:NH3 equilibrium}
\ce{NH4+} \leftrightarrow \ce{NH3} + \ce{H+}
\end{equation}

\begin{equation} \label{eq:pNH3}
\text{pNH}_3 = -\log \left([\text{NH}_3]_{\text{tot}}\right) + \log \left(1 + 10^{pK_a - pH} \right), \quad pK_a = 9.26 \text{ \cite{NationalCenterforBiotechnologyInformation2025PubChemAmmonia}}
\end{equation}

Similarly, anionic glycine and \ce{CN-} concentrations were calculated using \ref{eq:pGly} and \ref{eq:pCN}, respectively:
\begin{align} \label{eq:pGly}
\text{pGly} &= -\log \left([\text{Gly}_{\text{total}}]\right) + (pK_{a_1} - pH) \nonumber \\
&\quad + \log \left(1 + 10^{pH - pK_{a_1}} + \frac{1}{10^{pH - pK_{a_2}}} \right), \nonumber \\
&\quad pK_{a_1} = 2.37, \quad pK_{a_2} = 9.8 \text{ \cite{2025PubChemGlycine.}}
\end{align}
\begin{equation} \label{eq:pCN}
\text{pCN}^- = -\log \left([\text{CN}^-]_{\text{tot}}\right) + \log \left(1 + 10^{pK_a - pH} \right), \quad pK_a = 9.2 \text{ \cite{USEPA1980AmbientCyanides}}
\end{equation}
\begin{equation} \label{eq:pligand}
\text{pligand} = -\log[\text{ligand}].
\end{equation}

The interdependence of pH and ligand concentration (e.g., due to \ce{NH3} equilibrium in \ref{eq:NH3 equilibrium}) was modeled using these equations, which also include the assumption that ligand concentrations are unaffected by other ligands or external factors. For example, when both \ce{NH3} and \ce{CN-} were present, their equilibrium concentrations were treated as independent functions of pH.


Although cyanide is not deliberately introduced in EWAS systems, it is included in selected Pourbaix analyses to provide a conservative estimate of metal stability under conditions relevant to EWAS. Cyanide may arise as a transient intermediate during the electrochemical oxidation of organic nitrogen species \cite{Oraby2020GoldPermanganate, Chen2013AdsorptionStudy, Huerta1997ElectrochemicalPt111, Sandoval2011AdsorptionStudy}, and prior spectroscopic studies have shown that \ce{CN^-} can be electrochemically oxidized to cyanate under certain conditions \cite{Paulissen1992InfraredConditions, Hinman1986FourierElectrodes, Kitamura1986OxidationSpectroscopy, Chen2013AdsorptionStudy}.
As a strong-field ligand with a high affinity for transition metals, cyanide can significantly influence metal speciation and dissolution behavior. Moreover, due to its extreme toxicity and environmental impact \cite{xing2018simple, mekuto2016integrated, bruger2018volatilisation}, considering \ce{CN^-} in the Pourbaix framework provides a precautionary evaluation of electrode stability, particularly in worst-case scenarios involving ligand-induced leaching.


\section{Results and discussion}



\subsection{Thermodynamic screening of alloys} \label{sec:alloy_screening}





\section{Conclusion}

This study evaluates the stability of various transitional metal electrodes in EWAS conditions using Pourbaix diagrams, highlighting the critical role of nitrogen-containing ligands in governing corrosion behavior. While Ni and Cu exhibit strong resistance to oxidation in pure aqueous environments, they are highly susceptible to dissolution in the presence of \ce{CN^-} ligands. Pd and Pt are relatively inert toward glycine and ammonia but form stable cyanide complexes even at low concentrations, with inconsistencies in reported stability constants introducing uncertainty in their predicted durability. Given these limitations, alongside their high cost, Pd and Pt are less favorable pure metal candidates for EWAS deployment.

In contrast, Ti demonstrates excellent corrosion resistance across the EWAS-relevant pH and potential range due to its robust passivating oxide layer. However, the \ce{TiO2} layer's poor electronic conductivity may limit catalytic performance unless incorporated into conductive materials such as doped perovskites, MXenes, or metal alloys.

To address this, we explored the stability of Ti-based and Pd-based alloys using Pourbaix diagrams. Alloys such as Ni-Ti and Cu-Ti combine the corrosion resistance of Ti with the catalytic activity or conductivity of the dopant metal. Additionally, Au-Pd and Pt-Pd alloys are investigated for their potential to improve the balance between catalytic activity and stability under redox-dynamic EWAS conditions. These alloying strategies demonstrate that tailored materials design may mitigate the tradeoffs between stability and activity.

Overall, this work emphasizes the importance of integrating thermodynamic screening of both pure and alloyed materials to guide the rational design of EWAS electrodes with enhanced corrosion resistance, minimal leaching, and sustained performance in nitrogen-rich environments.


\section{CRediT authorship contribution statement}
Nianhan Tian: Conceptualization, Methodology, Formal analysis, Software, Visualization, Writing – Original Draft.  \\
Ehsan Abbasi: Investigation, Validation, Writing – Review \& Editing.  \\
Haldrian Iriawan: Investigation, Validation, Writing – Review \& Editing.  \\
Paul Kohl: Resources, Writing – Review \& Editing.  \\
Andrew J. Medford: Conceptualization, Supervision, Project administration, Writing – Review \& Editing.\\




\begin{acknowledgement}

% Please use ``The authors thank \ldots'' rather than ``The
% authors would like to thank \ldots''.

% The author thanks Mats Dahlgren for version one of \textsf{achemso},
% and Donald Arseneau for the code taken from \textsf{cite} to move
% citations after punctuation. Many users have provided feedback on the
% class, which is reflected in all of the different demonstrations
% shown in this document.

\end{acknowledgement}

%%%%%%%%%%%%%%%%%%%%%%%%%%%%%%%%%%%%%%%%%%%%%%%%%%%%%%%%%%%%%%%%%%%%%
%% The same is true for Supporting Information, which should use the
%% suppinfo environment.
%%%%%%%%%%%%%%%%%%%%%%%%%%%%%%%%%%%%%%%%%%%%%%%%%%%%%%%%%%%%%%%%%%%%%
\begin{suppinfo}


\end{suppinfo}

% \documentclass[journal=jacsat,manuscript=article,email=false]{achemso}

\usepackage[version=4]{mhchem} % Formula subscripts using \ce{}
\usepackage{esint}
\usepackage{threeparttable}
\usepackage{xr-hyper} % hyperlinks
\usepackage{hyperref} % hyperlinks
\usepackage{cleveref}
\usepackage{fancyhdr}
\usepackage[numbers]{natbib}
% To number sections, pages, figures and tables nested within chapters:
% \renewcommand{\thepage}{\arabic{chapter}.\arabic{page}} 
% \renewcommand{\thesection}{\arabic{chapter}.\arabic{section}}  
% \renewcommand{\thetable}{\arabic{chapter}.\arabic{table}}  
% \renewcommand{\thefigure}{\arabic{chapter}.\arabic{figure}}

% To number supplemental material with 'S':
% \renewcommand{\thepage}{S\arabic{page}} 
\renewcommand{\thesection}{S\arabic{section}}  
\renewcommand{\thetable}{S\arabic{table}}  
\renewcommand{\thefigure}{S\arabic{figure}}
\renewcommand{\theequation}{S\arabic{equation}}

\usepackage{longtable} % For multi-page tables
\usepackage{array}

\pagestyle{fancy}
\fancyhf{}
\cfoot{S\thepage}
\renewcommand{\headrulewidth}{0pt}


\author{Nianhan Tian}
\affiliation[Georgia Institute of Technology]
{School of Chemical and Biomolecular Engineering, Georgia Institute of Technology, Atlanta, Georgia 30318 USA}

\author{Andrew J. Medford}
\affiliation[Georgia Institute of Technology]
{School of Chemical and Biomolecular Engineering, Georgia Institute of Technology, Atlanta, Georgia 30318 USA}
\email{ajm@gatech.edu}

\title{Accelerated Computational Materials Discovery for Electrochemical Nutrient Recovery\\\vspace{8pt}\large{Supporting Information}}


\begin{document}
\newpage
\clearpage
\begin{longtable}{|p{4cm}|p{4cm}|p{3cm}|p{3cm}|}
\caption{Formation energies of species for \ce{NH3} complexes.} 
\label{tab:NH3_complex_energies}
\\
\hline
\textbf{Species} & \textbf{Metal ion} & \textbf{\( \Delta G^\circ_{298} \) (eV)} & \textbf{Reference} \\ \hline
\endfirsthead
\caption*{Table \thetable\ continued from previous pages.} \\
\hline
\textbf{Species} & \textbf{Metal ion} & \textbf{\( \Delta G^\circ_{298} \) (eV)} & \textbf{Reference} \\ \hline
\endhead
\hline
\endfoot
\hline
\endlastfoot
\ce{[Ag.0(NH3).0].0+} & \ce{Ag^1+} & 0.322 & \textnormal{\citenum{Bjerrum1957StabilitySubstances}} \\ \hline
\ce{[Ag.0(NH3)2.0].0+} & \ce{Ag^1+} & -0.191 & \textnormal{\citenum{Bjerrum1957StabilitySubstances}} \\ \hline
\ce{[Au.0(NH3)2.0].0+} & \ce{Au^1+} & -0.325 & \textnormal{\citenum{Bjerrum1957StabilitySubstances}} \\ \hline
\ce{[Au.0(NH3)4.0]^3.0+} & \ce{Au^3+} & 1.681 & \textnormal{\citenum{Bjerrum1957StabilitySubstances}} \\ \hline
\ce{[Ca.0(NH3).0]^2.0+} & \ce{Ca^2+} & -6.001 & \textnormal{\citenum{Bjerrum1957StabilitySubstances}} \\ \hline
\ce{[Ca.0(NH3)2.0]^2.0+} & \ce{Ca^2+} & -6.242 & \textnormal{\citenum{Bjerrum1957StabilitySubstances}} \\ \hline
\ce{[Ca.0(NH3)3.0]^2.0+} & \ce{Ca^2+} & -6.470 & \textnormal{\citenum{Bjerrum1957StabilitySubstances}} \\ \hline
\ce{[Ca.0(NH3)4.0]^2.0+} & \ce{Ca^2+} & -7.001 & \textnormal{\citenum{Bjerrum1957StabilitySubstances}} \\ \hline
\ce{[Cd.0(NH3).0]^2.0+} & \ce{Cd^2+} & -1.234 & \textnormal{\citenum{Bjerrum1957StabilitySubstances}} \\ \hline
\ce{[Cd.0(NH3)2.0]^2.0+} & \ce{Cd^2+} & -1.632 & \textnormal{\citenum{Bjerrum1957StabilitySubstances}} \\ \hline
\ce{[Cd.0(NH3)3.0]^2.0+} & \ce{Cd^2+} & -1.990 & \textnormal{\citenum{Bjerrum1957StabilitySubstances}} \\ \hline
\ce{[Cd.0(NH3)4.0]^2.0+} & \ce{Cd^2+} & -2.318 & \textnormal{\citenum{Bjerrum1957StabilitySubstances}} \\ \hline
\ce{[Cd.0(NH3)5.0]^2.0+} & \ce{Cd^2+} & -2.575 & \textnormal{\citenum{Bjerrum1957StabilitySubstances}} \\ \hline
\ce{[Cd.0(NH3)6.0]^2.0+} & \ce{Cd^2+} & -2.750 & \textnormal{\citenum{Bjerrum1957StabilitySubstances}} \\ \hline
\ce{[Co.0(NH3).0]^2.0+} & \ce{Co^2+} & -0.961 & \textnormal{\citenum{Bjerrum1957StabilitySubstances}} \\ \hline
\ce{[Co.0(NH3)2.0]^2.0+} & \ce{Co^2+} & -1.330 & \textnormal{\citenum{Bjerrum1957StabilitySubstances}} \\ \hline
\ce{[Co.0(NH3)3.0]^2.0+} & \ce{Co^2+} & -1.665 & \textnormal{\citenum{Bjerrum1957StabilitySubstances}} \\ \hline
\ce{[Co.0(NH3)4.0]^2.0+} & \ce{Co^2+} & -1.982 & \textnormal{\citenum{Bjerrum1957StabilitySubstances}} \\ \hline
\ce{[Co.0(NH3)5.0]^2.0+} & \ce{Co^2+} & -2.260 & \textnormal{\citenum{Bjerrum1957StabilitySubstances}} \\ \hline
\ce{[Co.0(NH3)6.0]^2.0+} & \ce{Co^2+} & -2.501 & \textnormal{\citenum{Bjerrum1957StabilitySubstances}} \\ \hline
\ce{[Co.0(NH3).0]^3.0+} & \ce{Co^3+} & 0.681 & \textnormal{\citenum{Bjerrum1957StabilitySubstances}} \\ \hline
\ce{[Co.0(NH3)2.0]^3.0+} & \ce{Co^3+} & 0.008 & \textnormal{\citenum{Bjerrum1957StabilitySubstances}} \\ \hline
\ce{[Co.0(NH3)3.0]^3.0+} & \ce{Co^3+} & -0.628 & \textnormal{\citenum{Bjerrum1957StabilitySubstances}} \\ \hline
\ce{[Co.0(NH3)4.0]^3.0+} & \ce{Co^3+} & -1.236 & \textnormal{\citenum{Bjerrum1957StabilitySubstances}} \\ \hline
\ce{[Co.0(NH3)5.0]^3.0+} & \ce{Co^3+} & -1.814 & \textnormal{\citenum{Bjerrum1957StabilitySubstances}} \\ \hline
\ce{[Co.0(NH3)6.0]^3.0+} & \ce{Co^3+} & -2.350 & \textnormal{\citenum{Bjerrum1957StabilitySubstances}} \\ \hline
\ce{[Cu.0(NH3).0].0+} & \ce{Cu^1+} & -0.107 & \textnormal{\citenum{Bjerrum1957StabilitySubstances}} \\ \hline
\ce{[Cu.0(NH3)2.0].0+} & \ce{Cu^1+} & -0.673 & \textnormal{\citenum{Bjerrum1957StabilitySubstances}} \\ \hline
\ce{[Cu.0(NH3).0]^2.0+} & \ce{Cu^2+} & 0.158 & \textnormal{\citenum{Bjerrum1957StabilitySubstances}} \\ \hline
\ce{[Cu.0(NH3)2.0]^2.0+} & \ce{Cu^2+} & -0.324 & \textnormal{\citenum{Bjerrum1957StabilitySubstances}} \\ \hline
\ce{[Cu.0(NH3)3.0]^2.0+} & \ce{Cu^2+} & -0.769 & \textnormal{\citenum{Bjerrum1957StabilitySubstances}} \\ \hline
\ce{[Cu.0(NH3)4.0]^2.0+} & \ce{Cu^2+} & -1.170 & \textnormal{\citenum{Bjerrum1957StabilitySubstances}} \\ \hline
\ce{[Fe.0(NH3).0]^2.0+} & \ce{Fe^2+} & -1.177 & \textnormal{\citenum{Bjerrum1957StabilitySubstances}} \\ \hline
\ce{[Fe.0(NH3)2.0]^2.0+} & \ce{Fe^2+} & -1.500 & \textnormal{\citenum{Bjerrum1957StabilitySubstances}} \\ \hline
\ce{[Fe.0(NH3)4.0]^2.0+} & \ce{Fe^2+} & -2.141 & \textnormal{\citenum{Bjerrum1957StabilitySubstances}} \\ \hline
\ce{[Ni.0(NH3).0]^2.0+} & \ce{Ni^2+} & -0.920 & \textnormal{\citenum{Bjerrum1957StabilitySubstances}} \\ \hline
\ce{[Ni.0(NH3)2.0]^2.0+} & \ce{Ni^2+} & -1.326 & \textnormal{\citenum{Bjerrum1957StabilitySubstances}} \\ \hline
\ce{[Ni.0(NH3)3.0]^2.0+} & \ce{Ni^2+} & -1.702 & \textnormal{\citenum{Bjerrum1957StabilitySubstances}} \\ \hline
\ce{[Ni.0(NH3)4.0]^2.0+} & \ce{Ni^2+} & -2.046 & \textnormal{\citenum{Bjerrum1957StabilitySubstances}} \\ \hline
\ce{[Ni.0(NH3)5.0]^2.0+} & \ce{Ni^2+} & -2.364 & \textnormal{\citenum{Bjerrum1957StabilitySubstances}} \\ \hline
\ce{[Ni.0(NH3)6.0]^2.0+} & \ce{Ni^2+} & -2.639 & \textnormal{\citenum{Bjerrum1957StabilitySubstances}} \\ \hline
\ce{[Mg.0(NH3).0]^2.0+} & \ce{Mg^2+} & -5.003 & \textnormal{\citenum{Bjerrum1957StabilitySubstances}} \\ \hline
\ce{[Mg.0(NH3)2.0]^2.0+} & \ce{Mg^2+} & -5.270 & \textnormal{\citenum{Bjerrum1957StabilitySubstances}} \\ \hline
\ce{[Mg.0(NH3)3.0]^2.0+} & \ce{Mg^2+} & -5.520 & \textnormal{\citenum{Bjerrum1957StabilitySubstances}} \\ \hline
\ce{[Mg.0(NH3)4.0]^2.0+} & \ce{Mg^2+} & -5.753 & \textnormal{\citenum{Bjerrum1957StabilitySubstances}} \\ \hline
\ce{[Mn.0(NH3).0]^2.0+} & \ce{Mn^2+} & -2.687 & \textnormal{\citenum{Bjerrum1957StabilitySubstances}} \\ \hline
\ce{[Mn.0(NH3)2.0]^2.0+} & \ce{Mn^2+} & -2.993 & \textnormal{\citenum{Bjerrum1957StabilitySubstances}} \\ \hline
\ce{[Zn.0(NH3).0]^2.0+} & \ce{Zn^2+} & -1.934 & \textnormal{\citenum{Bjerrum1957StabilitySubstances}} \\ \hline
\ce{[Zn.0(NH3)2.0]^2.0+} & \ce{Zn^2+} & -2.349 & \textnormal{\citenum{Bjerrum1957StabilitySubstances}} \\ \hline
\ce{[Zn.0(NH3)3.0]^2.0+} & \ce{Zn^2+} & -2.767 & \textnormal{\citenum{Bjerrum1957StabilitySubstances}} \\ \hline
\ce{[Zn.0(NH3)4.0]^2.0+} & \ce{Zn^2+} & -3.164 & \textnormal{\citenum{Bjerrum1957StabilitySubstances}} \\ \hline
\ce{[Pt.0(NH3)4.0]^2.0+} & \ce{Pt^2+} & -0.552 & \textnormal{\citenum{Sillen1964StabilityComplexes}} \\ \hline
\ce{[Pd.0(NH3).0]^2.0+} & \ce{Pd^2+} & 0.985 & \textnormal{\citenum{Smith1989CriticalConstants}} \\ \hline
\ce{[Pd.0(NH3)2.0]^2.0+} & \ce{Pd^2+} & 0.183 & \textnormal{\citenum{Smith1989CriticalConstants}} \\ \hline
\ce{[Pd.0(NH3)3.0]^2.0+} & \ce{Pd^2+} & -0.537 & \textnormal{\citenum{Smith1989CriticalConstants}} \\ \hline
\ce{[Pd.0(NH3)4.0]^2.0+} & \ce{Pd^2+} & -1.215 & \textnormal{\citenum{Smith1989CriticalConstants}} \\ \hline
\ce{[Zr.0(NH3).0].0+} & \ce{Zr^1+} & 3.127 & \textnormal{\citenum{Aviles2022ExploringNH3}} \\ \hline
\ce{[Zr.0(NH3)2.0].0+} & \ce{Zr^1+} & 3.248 & \textnormal{\citenum{Aviles2022ExploringNH3}} \\ \hline
\ce{[Zr.0(NH3)3.0].0+} & \ce{Zr^1+} & 2.498 & \textnormal{\citenum{Aviles2022ExploringNH3}} \\ \hline
\ce{[Zr.0(NH3)4.0].0+} & \ce{Zr^1+} & 2.450 & \textnormal{\citenum{Aviles2022ExploringNH3}} \\ \hline
\ce{[Zr.0(NH3)5.0].0+} & \ce{Zr^1+} & 2.042 & \textnormal{\citenum{Aviles2022ExploringNH3}} \\ \hline
\ce{[Zr.0(NH3)6.0].0+} & \ce{Zr^1+} & 2.307 & \textnormal{\citenum{Aviles2022ExploringNH3}} \\ \hline
\ce{[Zr.0(NH3)7.0].0+} & \ce{Zr^1+} & 1.930 & \textnormal{\citenum{Aviles2022ExploringNH3}}\end{longtable}

\newpage
\clearpage
\begin{longtable}{|p{4cm}|p{4cm}|p{3cm}|p{3cm}|}
\caption{Formation energies of species for \ce{Gly-} complexes.} 
\label{tab:Gly[1-]_complex_energies}
\\
\hline
\textbf{Species} & \textbf{Metal ion} & \textbf{\( \Delta G^\circ_{298} \) (eV)} & \textbf{Reference} \\ \hline
\endfirsthead
\caption*{Table \thetable\ continued from previous pages.} \\
\hline
\textbf{Species} & \textbf{Metal ion} & \textbf{\( \Delta G^\circ_{298} \) (eV)} & \textbf{Reference} \\ \hline
\endhead
\hline
\endfoot
\hline
\endlastfoot
\ce{[Au(Gly)2]-} & \ce{Au^1+} & -5.613 & \textnormal{\citenum{Azadi2019DataComplexes}} \\ \hline
\ce{[Au(Gly)]^2+} & \ce{Au^3+} & 0.880 & \textnormal{\citenum{Azadi2019DataComplexes}} \\ \hline
\ce{[Au(Gly)2]+} & \ce{Au^3+} & -2.591 & \textnormal{\citenum{Azadi2019DataComplexes}} \\ \hline
\ce{[Ag(Gly)]} & \ce{Ag^1+} & -3.046 & \textnormal{\citenum{Smith1989CriticalConstants}} \\ \hline
\ce{[Ag(Gly)2]-} & \ce{Ag^1+} & -6.458 & \textnormal{\citenum{Smith1989CriticalConstants}} \\ \hline
\ce{[Fe(Gly)]+} & \ce{Fe^2+} & -4.335 & \textnormal{\citenum{Smith1989CriticalConstants}} \\ \hline
\ce{[Fe(Gly)2]} & \ce{Fe^2+} & -7.805 & \textnormal{\citenum{Smith1989CriticalConstants}} \\ \hline
\ce{[Fe(Gly)]^2+} & \ce{Fe^3+} & -3.785 & \textnormal{\citenum{Smith1989CriticalConstants}} \\ \hline
\ce{[Cu(Gly)]} & \ce{Cu^1+} & -3.144 & \textnormal{\citenum{Smith1989CriticalConstants}} \\ \hline
\ce{[Cu(Gly)2]-} & \ce{Cu^1+} & -6.600 & \textnormal{\citenum{Smith1989CriticalConstants}} \\ \hline
\ce{[Cu(Gly)]+} & \ce{Cu^2+} & -3.064 & \textnormal{\citenum{Smith1989CriticalConstants}} \\ \hline
\ce{[Cu(Gly)2]} & \ce{Cu^2+} & -6.740 & \textnormal{\citenum{Smith1989CriticalConstants}} \\ \hline
\ce{[Cu(Gly)3]-} & \ce{Cu^2+} & -9.110 & \textnormal{\citenum{Smith1989CriticalConstants}} \\ \hline
\ce{[Mg(Gly)]+} & \ce{Mg^2+} & -8.180 & \textnormal{\citenum{Smith1989CriticalConstants}} \\ \hline
\ce{[Mg(Gly)2]} & \ce{Mg^2+} & -11.240 & \textnormal{\citenum{Smith1989CriticalConstants}} \\ \hline
\ce{[Mn(Gly)]+} & \ce{Mn^2+} & -5.816 & \textnormal{\citenum{Smith1989CriticalConstants}} \\ \hline
\ce{[Mn(Gly)2]} & \ce{Mn^2+} & -9.216 & \textnormal{\citenum{Smith1989CriticalConstants}} \\ \hline
\ce{[Mn(Gly)3]-} & \ce{Mn^2+} & -12.479 & \textnormal{\citenum{Smith1989CriticalConstants}} \\ \hline
\ce{[Ni(Gly)]+} & \ce{Ni^2+} & -4.087 & \textnormal{\citenum{Smith1989CriticalConstants}} \\ \hline
\ce{[Ni(Gly)2]} & \ce{Ni^2+} & -7.640 & \textnormal{\citenum{Smith1989CriticalConstants}} \\ \hline
\ce{[Ni(Gly)3]-} & \ce{Ni^2+} & -11.065 & \textnormal{\citenum{Smith1989CriticalConstants}} \\ \hline
\ce{[Zn(Gly)]+} & \ce{Zn^2+} & -5.114 & \textnormal{\citenum{Smith1989CriticalConstants}} \\ \hline
\ce{[Zn(Gly)2]} & \ce{Zn^2+} & -8.639 & \textnormal{\citenum{Smith1989CriticalConstants}} \\ \hline
\ce{[Zn(Gly)3]-} & \ce{Zn^2+} & -11.994 & \textnormal{\citenum{Smith1989CriticalConstants}} \\ \hline
\ce{[Co(Gly)]+} & \ce{Co^2+} & -4.105 & \textnormal{\citenum{Smith1989CriticalConstants}} \\ \hline
\ce{[Co(Gly)2]} & \ce{Co^2+} & -7.593 & \textnormal{\citenum{Smith1989CriticalConstants}} \\ \hline
\ce{[Co(Gly)3]-} & \ce{Co^2+} & -11.004 & \textnormal{\citenum{Smith1989CriticalConstants}} \\ \hline
\ce{[Cd(Gly)]+} & \ce{Cd^2+} & -4.312 & \textnormal{\citenum{Smith1989CriticalConstants}} \\ \hline
\ce{[Cd(Gly)2]} & \ce{Cd^2+} & -7.772 & \textnormal{\citenum{Smith1989CriticalConstants}} \\ \hline
\ce{[Cd(Gly)3]-} & \ce{Cd^2+} & -11.203 & \textnormal{\citenum{Azadi2019DataComplexes}} \\ \hline
\ce{[Na(Gly)]} & \ce{Na^1+} & -5.948 & \textnormal{\citenum{Azadi2019DataComplexes}} \\ \hline
\ce{[Ti(Gly)]} & \ce{Ti^1+} & -3.017 & \textnormal{\citenum{Azadi2019DataComplexes}} \\ \hline
\ce{[Ti(Gly)2]-} & \ce{Ti^1+} & -7.080 & \textnormal{\citenum{Azadi2019DataComplexes}} \\ \hline
\ce{[Ca(Gly)]+} & \ce{Ca^2+} & -9.085 & \textnormal{\citenum{Kiss1991CriticalGlycine}} \\ \hline
\ce{[Sr(Gly)]+} & \ce{Sr^2+} & -9.115 & \textnormal{\citenum{Kiss1991CriticalGlycine}} \\ \hline
\ce{[Pd(Gly)]+} & \ce{Pd^2+} & -2.016 & \textnormal{\citenum{Kiss1991CriticalGlycine}} \\ \hline
\ce{[Pd(Gly)2]} & \ce{Pd^2+} & -5.778 & \textnormal{\citenum{Smith1989CriticalConstants}}\end{longtable}

\newpage
\clearpage
\begin{longtable}{|p{4cm}|p{4cm}|p{3cm}|p{3cm}|}
\caption{Formation energies of species for \ce{CN-} complexes.} 
\label{tab:CN-_complex_energies}
\\
\hline
\textbf{Species} & \textbf{Metal ion} & \textbf{\( \Delta G^\circ_{298} \) (eV)} & \textbf{Reference} \\ \hline
\endfirsthead
\caption*{Table \thetable\ continued from previous pages.} \\
\hline
\textbf{Species} & \textbf{Metal ion} & \textbf{\( \Delta G^\circ_{298} \) (eV)} & \textbf{Reference} \\ \hline
\endhead
\hline
\endfoot
\hline
\endlastfoot
\ce{[Ag(CN)2]-} & \ce{Ag^1+} & 3.124 & \textnormal{\citenum{Beck1987CriticalComplexes}} \\ \hline
\ce{[Ag(CN)3]^2-} & \ce{Ag^1+} & 4.864 & \textnormal{\citenum{Beck1987CriticalComplexes}} \\ \hline
\ce{[Ag(CN)4]^3-} & \ce{Ag^1+} & 6.722 & \textnormal{\citenum{Beck1987CriticalComplexes}} \\ \hline
\ce{[Au(CN)2]-} & \ce{Au^1+} & 3.132 & \textnormal{\citenum{Beck1987CriticalComplexes}} \\ \hline
\ce{[Au(CN)4]-} & \ce{Au^3+} & 8.394 & \textnormal{\citenum{Beck1987CriticalComplexes}} \\ \hline
\ce{[Cd(CN)]+} & \ce{Cd^2+} & 0.657 & \textnormal{\citenum{Beck1987CriticalComplexes}} \\ \hline
\ce{[Cd(CN)2]} & \ce{Cd^2+} & 2.142 & \textnormal{\citenum{Beck1987CriticalComplexes}} \\ \hline
\ce{[Cd(CN)3]-} & \ce{Cd^2+} & 3.651 & \textnormal{\citenum{Beck1987CriticalComplexes}} \\ \hline
\ce{[Cd(CN)4]^2-} & \ce{Cd^2+} & 5.225 & \textnormal{\citenum{Beck1987CriticalComplexes}} \\ \hline
\ce{[Cu(CN)2]-} & \ce{Cu^1+} & 2.672 & \textnormal{\citenum{Beck1987CriticalComplexes}} \\ \hline
\ce{[Cu(CN)3]^2-} & \ce{Cu^1+} & 4.186 & \textnormal{\citenum{Beck1987CriticalComplexes}} \\ \hline
\ce{[Cu(CN)4]^3-} & \ce{Cu^1+} & 5.872 & \textnormal{\citenum{Beck1987CriticalComplexes}} \\ \hline
\ce{[Fe(CN)6]^4-} & \ce{Fe^2+} & 7.809 & \textnormal{\citenum{Beck1987CriticalComplexes}} \\ \hline
\ce{[Fe(CN)6]^3-} & \ce{Fe^3+} & 8.092 & \textnormal{\citenum{Beck1987CriticalComplexes}} \\ \hline
\ce{[Ni(CN)4]^2-} & \ce{Ni^2+} & 4.814 & \textnormal{\citenum{Beck1987CriticalComplexes}} \\ \hline
\ce{[Pd(CN)]+} & \ce{Pd^2+} & 2.995 & \textnormal{\citenum{Sillen1964StabilityComplexes}} \\ \hline
\ce{[Pd(CN)4]^2-} & \ce{Pd^2+} & 6.468 & \textnormal{\citenum{Beck1987CriticalComplexes}} \\ \hline
\ce{[Pd(CN)5]^3-} & \ce{Pd^2+} & 8.083 & \textnormal{\citenum{Beck1987CriticalComplexes}} \\ \hline
\ce{[Zn(CN)4]^2-} & \ce{Zn^2+} & 4.634 & \textnormal{\citenum{Beck1987CriticalComplexes}} \\ \hline
\ce{[Pt(CN)4]^2-} & \ce{Pt^2+} & -7.364 & \textnormal{\citenum{Sillen1964StabilityComplexes}}\end{longtable}

\newpage
\clearpage
\begin{longtable}{|p{4cm}|p{3cm}|p{3cm}|}
\caption{Formation energies of Fe species queried from Materials Project\cite{Jain2013TheInnovation}.} 
\label{tab:bulk_Fe_energies}
\\
\hline
\textbf{Species}  & \textbf{State} & \textbf{\( \Delta G\) (eV)} \\ \hline
\endfirsthead
\caption*{Table \thetable\ continued from previous pages.} \\
\hline
\textbf{Species}  & \textbf{State} & \textbf{\( \Delta G\) (eV)} \\ \hline
\endhead
\hline
\endfoot
\hline
\endlastfoot
\ce{FeO2^2-} & Ion & -3.011 \\ \hline
\ce{FeOH+} & Ion & -2.824 \\ \hline
\ce{Fe(OH)3} & Ion & -6.784 \\ \hline
\ce{FeOH^2+} & Ion & -4.743 \\ \hline
\ce{FeO4^2-} & Ion & -3.290 \\ \hline
\ce{Fe^2+} & Ion & -0.768 \\ \hline
\ce{Fe^3+} & Ion & 0.002 \\ \hline
\ce{Fe(OH)2+} & Ion & -4.490 \\ \hline
\ce{FeO2-} & Ion & -3.767 \\ \hline
\ce{FeHO2-} & Ion & -3.866 \\ \hline
\ce{Fe100} & Solid & 27.946 \\ \hline
\ce{Fe28} & Solid & 4.908 \\ \hline
\ce{Fe} & Solid & 0.000 \\ \hline
\ce{Fe4} & Solid & 0.068 \\ \hline
\ce{Fe2} & Solid & 0.196 \\ \hline
\ce{Fe6H2} & Solid & 5.815 \\ \hline
\ce{Fe3H} & Solid & 2.917 \\ \hline
\ce{FeH3} & Solid & 2.517 \\ \hline
\ce{Fe2H8} & Solid & 8.721 \\ \hline
\ce{FeH} & Solid & 0.506 \\ \hline
\ce{Fe2H6} & Solid & 6.189 \\ \hline
\ce{Fe4H8O8} & Solid & -6.965 \\ \hline
\ce{Fe16H16O32} & Solid & -71.989 \\ \hline
\ce{FeH2O2} & Solid & -1.673 \\ \hline
\ce{Fe4H4O8} & Solid & -19.690 \\ \hline
\ce{Fe16H20O34} & Solid & -44.238 \\ \hline
\ce{Fe2H2O4} & Solid & -9.439 \\ \hline
\ce{Fe4H14O13} & Solid & -6.242 \\ \hline
\ce{Fe42H2O64} & Solid & -145.257 \\ \hline
\ce{Fe10H2O16} & Solid & -37.318 \\ \hline
\ce{Fe21HO32} & Solid & -70.594 \\ \hline
\ce{Fe14O15} & Solid & -37.651 \\ \hline
\ce{Fe15O16} & Solid & -39.621 \\ \hline
\ce{Fe13O14} & Solid & -34.772 \\ \hline
\ce{Fe4O4} & Solid & -10.585 \\ \hline
\ce{Fe11O12} & Solid & -30.279 \\ \hline
\ce{Fe5O7} & Solid & -15.783 \\ \hline
\ce{Fe13O19} & Solid & -38.495 \\ \hline
\ce{FeO} & Solid & -0.541 \\ \hline
\ce{Fe12O12} & Solid & -29.265 \\ \hline
\ce{Fe32O48} & Solid & -99.453 \\ \hline
\ce{Fe2O6} & Solid & -2.312 \\ \hline
\ce{Fe40O40} & Solid & -83.627 \\ \hline
\ce{Fe12O13} & Solid & -31.988 \\ \hline
\ce{Fe38O39} & Solid & -97.393 \\ \hline
\ce{Fe23O25} & Solid & -63.074 \\ \hline
\ce{Fe2O2} & Solid & -5.255 \\ \hline
\ce{Fe35O36} & Solid & -39.295 \\ \hline
\ce{Fe10O14} & Solid & -30.547 \\ \hline
\ce{Fe21O27} & Solid & -58.877 \\ \hline
\ce{Fe16O18} & Solid & -45.275 \\ \hline
\ce{Fe21O32} & Solid & -69.168 \\ \hline
\ce{Fe8O9} & Solid & -22.422 \\ \hline
\ce{Fe9O10} & Solid & -25.373 \\ \hline
\ce{Fe64O96} & Solid & -223.631 \\ \hline
\ce{Fe16O24} & Solid & -57.830 \\ \hline
\ce{Fe5O8} & Solid & -16.442 \\ \hline
\ce{Fe23O32} & Solid & -75.610 \\ \hline
\ce{Fe4O6} & Solid & -15.171 \\ \hline
\ce{Fe7O8} & Solid & -19.348 \\ \hline
\ce{Fe21O23} & Solid & -57.856 \\ \hline
\ce{Fe32O35} & Solid & -87.753 \\ \hline
\ce{Fe8O12} & Solid & -28.623 \\ \hline
\ce{Fe41O56} & Solid & -134.661 \\ \hline
\ce{Fe2O3} & Solid & -3.287 \\ \hline
\ce{Fe24O32} & Solid & -79.283 \\ \hline
\ce{Fe12O16} & Solid & -40.319 \\ \hline
\ce{Fe12O18} & Solid & -30.420 \\ \hline
\ce{Fe6O8} & Solid & -20.499 \\ \hline
\ce{Fe3O4} & Solid & -9.380 \\ \hline
\ce{Fe9O13} & Solid & -27.197 \\ \hline
\ce{Fe20O32} & Solid & -66.720 \\ \hline
\ce{Fe6O2} & Solid & 6.795 \\ \hline
\ce{Fe2O4} & Solid & -6.397 \\ \hline
\ce{Fe4O8} & Solid & -12.379 \\ \hline
\ce{Fe17O18} & Solid & -45.891 \\ \hline
\ce{Fe20O22} & Solid & -55.652 \\ \hline
\ce{Fe10O11} & Solid & -27.581 \\ \hline
\ce{Fe12O24} & Solid & -33.858 \\ \hline
\ce{Fe43O64} & Solid & -149.513 \\ \hline
\ce{Fe16O34} & Solid & -44.989 \\ \hline
\ce{Fe14O16} & Solid & -39.748 \\ \hline
\ce{Fe13O15} & Solid & -36.844 \\ \hline
\ce{Fe4O13} & Solid & -3.095 \\ \hline
\ce{Fe8O16} & Solid & -24.749 \\ \hline
\ce{FeO2} & Solid & -2.864 \\ \hline
\ce{Fe8O20} & Solid & -17.319 \\ \hline
\ce{Fe8O10} & Solid & -24.179 \\ \hline
\ce{Fe25O32} & Solid & -71.771 \\ \hline
\ce{Fe16O32} & Solid & -42.120\end{longtable}
\bibliography{references_mendeley_pourbaix}

\end{document}
%%%%%%%%%%%%%%%%%%%%%%%%%%%%%%%%%%%%%%%%%%%%%%%%%%%%%%%%%%%%%%%%%%%%%
%% The appropriate \bibliography command should be placed here.
%% Notice that the class file automatically sets \bibliographystyle
%% and also names the section correctly.
%%%%%%%%%%%%%%%%%%%%%%%%%%%%%%%%%%%%%%%%%%%%%%%%%%%%%%%%%%%%%%%%%%%%%
\bibliography{references_mendeley_pourbaix, references}

\end{document}